\documentclass{beamer}
\usepackage[utf8]{inputenc}

\usepackage{utopia} %font utopia imported
\setbeamercovered{transparent}

\usetheme{Warsaw}
% \usecolortheme{orchid}

%------------------------------------------------------------
%This block of code defines the information to appear in the
%Title page
\title[Proof of Novelty] %optional
{Proof of Novelty}

\subtitle{A distributed consensus mechanism for securing content novelty}

\author[Severo, Daniel] % (optional)
{Daniel Severo}

\institute % (optional)
{Independent Scientist}


\date[VDC 2019] % (optional)
{Virtual Design Challenge \\ The University of British Columbia \\ Vancouver - December 2019}

%End of title page configuration block
%------------------------------------------------------------



%------------------------------------------------------------
%The next block of commands puts the table of contents at the 
%beginning of each section and highlights the current section:

\AtBeginSection[]
{
  \begin{frame}
    \frametitle{Table of Contents}
    \tableofcontents[currentsection]
  \end{frame}
}
%------------------------------------------------------------


\begin{document}

%The next statement creates the title page.
\frame{\titlepage}


%---------------------------------------------------------
%This block of code is for the table of contents after
%the title page
\begin{frame}
\frametitle{Table of Contents}
\tableofcontents
\end{frame}
%---------------------------------------------------------


\section{Motivation}
\begin{frame}{Motivation}
    \begin{block}{Motivational question}
        Somebody sends you a video, how do you know it is trustworthy?
    \end{block}
    trustworthy = authentic + novel
\end{frame}

\section{Current Solutions}
\begin{frame}{Current Solutions}
\end{frame}

\section{Patent Systems}
\begin{frame}{Patent Systems}
\end{frame}

\section{Recap}
\begin{frame}{Recap}
\begin{enumerate}
    \item<1-> Trustworthy content must be authentic and novel.

    \item<2-> Authenticity can be solved with Digital Signatures, but novelty requires searching for Prior Art.
    
    \item<3-> Trustless and distributed Prior Art search can be done using Blockchains, Content-addressable storage, Cryptographic Sortition and Similarity Measures.

    \item<4-> 
\end{enumerate}
\end{frame}

% \begin{frame}{Frame Title}
% \begin{columns}
% \begin{column}{0.5\textwidth}
%   some text here some text here some text here some text here some text here
% \end{column}
% \begin{column}{0.5\textwidth}  %%<--- here
%     \begin{center}
%      \includegraphics[width=0.5\textwidth]{image1.jpg}
%      \end{center}
% \end{column}
% \end{columns}
% \end{frame}
%---------------------------------------------------------
%Changing visivility of the text
% \begin{frame}
% \frametitle{Sample frame title}
% This is a text in second frame. For the sake of showing an example.

% \begin{itemize}
%     \item<1-> Text visible on slide 1
%     \item<2-> Text visible on slide 2
%     \item<3> Text visible on slides 3
%     \item<4-> Text visible on slide 4
% \end{itemize}
% \end{frame}

%---------------------------------------------------------


%---------------------------------------------------------
% %Example of the \pause command
% \begin{frame}
% In this slide \pause

% the text will be partially visible \pause

% And finally everything will be there
% \end{frame}
%---------------------------------------------------------

% \section{Second section}

%---------------------------------------------------------
%Highlighting text
% \begin{frame}
% \frametitle{Sample frame title}

% In this slide, some important text will be
% \alert{highlighted} because it's important.
% Please, don't abuse it.

% \begin{block}{Remark}
% Sample text
% \end{block}

% \begin{alertblock}{Important theorem}
% Sample text in red box
% \end{alertblock}

% \begin{examples}
% Sample text in green box. The title of the block is ``Examples".
% \end{examples}
% \end{frame}
%---------------------------------------------------------


%---------------------------------------------------------
%Two columns
% \begin{frame}
% \frametitle{Two-column slide}

% \begin{columns}

% \column{0.5\textwidth}
% This is a text in first column.
% $$E=mc^2$$
% \begin{itemize}
% \item First item
% \item Second item
% \end{itemize}

% \column{0.5\textwidth}
% This text will be in the second column
% and on a second tought this is a nice looking
% layout in some cases.
% \end{columns}
% \end{frame}
% %---------------------------------------------------------


\end{document}