\documentclass[conference]{IEEEtran}
\IEEEoverridecommandlockouts
% The preceding line is only needed to identify funding in the first footnote. If that is unneeded, please comment it out.
\usepackage{cite}
\usepackage{amsmath,amssymb,amsfonts}
\usepackage{algorithmic}
\usepackage{graphicx}
\usepackage{textcomp}
\usepackage{xcolor}
\def\BibTeX{{\rm B\kern-.05em{\sc i\kern-.025em b}\kern-.08em
    T\kern-.1667em\lower.7ex\hbox{E}\kern-.125emX}}
\begin{document}

\title{UNITLED\\
\thanks{This paper was written for the VIRTUAL DESIGN CHALLENGE FOR AUTHENTICATING AND PROTECTING FULL MOTION VIDEOS: \url{https://blockchain.ubc.ca/news/virtual-design-challenge-authenticating-and-protecting-full-motion-videos}}
}

\author{\IEEEauthorblockN{Daniel Severo}
\IEEEauthorblockA{\textit{Independent Researcher} \\
São Paulo, Brazil \\
danielsouzasevero@gmail.com}
}

\maketitle

\begin{abstract}
UNTITLED is a Distributed Patent System design proposal equipped with a Prior Art search mechanism. Patentability requirements are defined through a consensus protocol using Smart Contracts and collaborative training of machine learning algorithms. Removing the patent clerk and shifting the burden of proof of patentability to the patentee, we guarantee scalability and efficiency of the blockchain. The system is highly sensitive to the invention’s medium (e.g. audio files, textual documents) as it defines the class of algorithms employed to calculate similarity, which in turn defines what constitutes as prior art. We show how UNTITLED can be applied to full-motion video archives to protect against tampering and dissemination of false information through Deepfakes.\end{abstract}

\begin{IEEEkeywords}
blockchain, patent, machine learning, smart contracts, prior art, full-motion videos
\end{IEEEkeywords}

\section{Introduction}
TODO\\
- contribution 1: distributed patent system with prior art search;\\
- contribution 2: secure media (e.g. FMVs) using patent grants as proof of novelty;

\section{Background}
\subsection{Patent Systems and Prior Art}
A patent is a legal document that provides proof of ownership of intellectual property. It is commonly issued by government run agencies to individuals or organizations. The procedure of emission is initiated with a formal request by the party interested in obtaining the patent, called a \emph{patentee}. A \emph{patent clerk}, representing the emitting agency, is then attributed with the task of verifying patentability conditions such as (but not limited to) novelty and non-obviousness of the invention. This is done by collecting evidence of absence of previous similar work, called \emph{Prior Art}. What constitutes as Prior Art is often the cause of discourse due to its subjective nature and the centralization of power on patent clerks and agencies.

TODO [patent grants as proof of novelty]

\subsection{Blockchains and Smart Contracts}
\subsection{Collaborative Machine Learning}
\section{Problem Statement}
\section{Related Work}
TODO \\
- naive systems\\
- ARCHANGEL: TCH needs overfitting; doesn't address discovery\\
- Bernstein\\
- Mediachain\\
- Locality-sensitive hashing\\
- https://arxiv.org/abs/1901.03136\\
\section{Design Proposal}
\section{Application to Full-motion Video Archives}
\section{Conclusions and Future Work}
\section*{Acknowledgment}
We would like to thank Professor Chen Feng for the invitation to participate in the VIRTUAL DESIGN CHALLENGE FOR AUTHENTICATING  AND  PROTECTING  FULL  MOTION  VIDEOS, hosted by Patriot One Technologies Inc. in collaboration with the Blockchain research cluster at The University of British Columbia.
\section*{References}
\end{document}
